\href{http://mqtt.org/}{\tt M\+Q\+TT} publish/subscribe library for Photon, Spark Core version 0.\+4.\+28.

\subsection*{Source Code}

This lightweight library source code are only 2 files. firmware -\/$>$ \hyperlink{_m_q_t_t_8cpp}{M\+Q\+T\+T.\+cpp}, M\+Q\+T\+T.\+h.

Application can use Q\+O\+S0,1,2 and retain flag when send a publish message.

\subsection*{Example}

Some sample sketches for Spark Core and Photon included(firmware/examples/).
\begin{DoxyItemize}
\item mqtttest.\+ino \+: simple pub/sub sample.
\item mqttqostest.\+ino \+: Qo\+S1, Qo\+S2 publish and callback sample.
\end{DoxyItemize}

\subsection*{developer examples}

some applications use \hyperlink{class_m_q_t_t}{M\+Q\+TT} with Photon. here are developer\textquotesingle{}s reference examples.
\begin{DoxyItemize}
\item \href{http://www.instructables.com/id/Spark-Core-Photon-and-CloudMQTT/}{\tt Spark Core / Photon and Cloud\+M\+Q\+TT}
\item \href{https://www.hackster.io/anasdalintakam/mqtt-publish-subscribe-using-rpi-esp-and-photon-864fe9#toc--particle-photon-as-mqtt-client-2}{\tt M\+Q\+TT Publish-\/\+Subscribe Using Rpi, E\+SP and Photon}
\item \href{http://www.kevinhoyt.com/2016/04/27/particle-photon-on-watson-iot/}{\tt Particle Photon on Watson IoT}
\item \href{https://developer.ibm.com/recipes/tutorials/connecting-a-iot-device-of-the-watson-conversation-cardashboard-app/}{\tt Connecting IoT devices to the Watson Conversation Car-\/\+Dashboard app}
\item \href{https://jp.mathworks.com/help/thingspeak/mqtt-api.html}{\tt Thing\+Speak M\+Q\+TT A\+PI}
\item \href{https://www.losant.com/blog/how-to-connect-a-particle-photon-to-the-losant-iot-platform}{\tt H\+OW TO C\+O\+N\+N\+E\+CT A P\+A\+R\+T\+I\+C\+LE P\+H\+O\+T\+ON TO T\+HE L\+O\+S\+A\+NT I\+OT P\+L\+A\+T\+F\+O\+RM}
\item \href{https://medium.com/@stevecaldwell/how-i-hacked-my-humidor-with-losant-and-a-particle-photon-84342744755b#.b68apdmo1}{\tt How I Hacked my Humidor with Losant and a Particle Photon}
\item \href{https://developer.artik.io/documentation/advanced-concepts/mqtt/color-mqtt.html}{\tt A\+R\+T\+IK as M\+Q\+TT Message Broker}
\item \href{https://ubidots.com/docs/devices/particleMQTT.html}{\tt Particle and Ubidots using M\+Q\+TT}
\item \href{https://www.twilio.com/docs/quickstart/sync-iot/mqtt-particle-photon-sync-iot}{\tt U\+S\+I\+NG T\+W\+I\+L\+IO S\+Y\+NC W\+I\+TH M\+Q\+TT ON A P\+A\+R\+T\+I\+C\+LE P\+H\+O\+T\+ON}
\end{DoxyItemize}

\#\# sample source 
\begin{DoxyCode}
#include "application.h"
#include "MQTT.h"

void callback(char* topic, byte* payload, unsigned int length);
MQTT client("iot.eclipse.org", 1883, callback);

// recieve message
void callback(char* topic, byte* payload, unsigned int length) \{
    char p[length + 1];
    memcpy(p, payload, length);
    p[length] = NULL;

    if (!strcmp(p, "RED"))
        RGB.color(255, 0, 0);
    else if (!strcmp(p, "GREEN"))
        RGB.color(0, 255, 0);
    else if (!strcmp(p, "BLUE"))
        RGB.color(0, 0, 255);
    else
        RGB.color(255, 255, 255);
    delay(1000);
\}


void setup() \{
    RGB.control(true);

    // connect to the server(unique id by Time.now())
    client.connect("sparkclient\_" + String(Time.now()));

    // publish/subscribe
    if (client.isConnected()) \{
        client.publish("outTopic/message","hello world");
        client.subscribe("inTopic/message");
    \}
\}

void loop() \{
    if (client.isConnected())
        client.loop();
\}
\end{DoxyCode}
 \subsection*{F\+AQ}

\subsubsection*{Can\textquotesingle{}t connect/publish/subscribe to the \hyperlink{class_m_q_t_t}{M\+Q\+TT} server?}


\begin{DoxyItemize}
\item Check your \hyperlink{class_m_q_t_t}{M\+Q\+TT} server and port(default 1883) is really working with the mosquitto\+\_\+pub/sub command. And maybe your \hyperlink{class_m_q_t_t}{M\+Q\+TT} server can\textquotesingle{}t connect from Internet because of firewall. Check your network environments.
\item Check your subscribe/publish topic name is really matched.
\item Perhaps device firmware network stack is failed. check your firmware version and bugs.
\item If you use M\+Q\+T\+T-\/\+T\+LS, check your RooT CA pem file, client key, certifications is okay or not.
\item Several \hyperlink{class_m_q_t_t}{M\+Q\+TT} server will disconnect to the 1st connection when you use the same user\+\_\+id. When the application call the connect method, use different user\+\_\+id in every devices in connect method\textquotesingle{}s 2nd argument. Use M\+AC address as a user\+\_\+id will be better. 
\begin{DoxyPre}
   // device.1
   client.connect("spark-client", "user\_1", "password1");
   // other devices...
   client.connect("spark-client", "user\_others", "password1");
\end{DoxyPre}

\end{DoxyItemize}

\subsubsection*{I want to change \hyperlink{class_m_q_t_t}{M\+Q\+TT} keep alive timeout.}

\hyperlink{class_m_q_t_t}{M\+Q\+TT} keepalive timeout is defined \char`\"{}\+M\+Q\+T\+T\+\_\+\+D\+E\+F\+A\+U\+L\+T\+\_\+\+K\+E\+E\+P\+A\+L\+I\+V\+E 15\char`\"{}(15 sec) in header file. You can change the keepalive timeout in constructor. 
\begin{DoxyPre}
    \hyperlink{class_m_q_t_t}{MQTT} client("server\_name", 1883, callback); // default: send keepalive packet to \hyperlink{class_m_q_t_t}{MQTT} server in every 15sec.
    \hyperlink{class_m_q_t_t}{MQTT} client("server\_name", 1883, 30, callback); // keepliave timeout is 30sec.
\end{DoxyPre}


\subsubsection*{Want to use over the 255 message size.}

In this library, max \hyperlink{class_m_q_t_t}{M\+Q\+TT} message size is defined \char`\"{}\+M\+Q\+T\+T\+\_\+\+M\+A\+X\+\_\+\+P\+A\+C\+K\+E\+T\+\_\+\+S\+I\+Z\+E 255\char`\"{} in header file. But If you want to use over 255bytes, use the constructor 4th argument. 
\begin{DoxyPre}
    \hyperlink{class_m_q_t_t}{MQTT} client("server\_name", 1883, callback); // default 255bytes
    \hyperlink{class_m_q_t_t}{MQTT} client("server\_name", 1883, MQTT\_DEFAULT\_KEEPALIVE, callback, 512); // max 512bytes
\end{DoxyPre}


\subsubsection*{Can I use on old firmware?}

No, use default latest firmware. I test this library on default latest firmware or latest pre-\/release version. If you really want to use old firmware(I think don\textquotesingle{}t need that case), maybe it can\textquotesingle{}t work well and it is out of my assumption.

\subsubsection*{Bug or Problem?}

First of all, check the \href{https://community.particle.io/}{\tt Particle community site. But still your problem will not clear, please send a bug-\/fixed diff and Pull request or problem details to issue. Pull Request If you have a bug or feature, please send a pull request. Thanks for all developer\textquotesingle{}s pull request! }